\documentclass[11pt,a4paper]{article}
\usepackage[top=1.00in, bottom=1.0in, left=1.1in, right=1.1in]{geometry}
\usepackage{graphicx}
\usepackage{natbib}
\usepackage[export]{adjustbox}

% http://www.nature.com/nclimate/authors/gta/ed-process/index.html

% Researchers should supply a brief paragraph stating the interest to a broad scientific readership, address and contact details, title, a fully referenced summary paragraph, and a list of the references cited in the summary paragraph. Additional material can be included as a separate file if needed.

\begin{document}

\noindent \includegraphics[width=0.4\textwidth, right]{/Users/Lizzie/Documents/Professional/images/letterhead/arnold/AASmLogo2colr.jpg}
\vspace{1ex}\\

\noindent Dear Dr. Hetherington: %Check?
\vspace{1.5ex}\\
\noindent Please consider our manuscript `Temperature and photoperiod drive spring phenology across all species in a temperate forest community' as a Full Paper for \emph{New Phytologist.} We present new experimental results that show temperate plant spring phenology is affected by three interactive cues, suggesting responses to continued warming will be complex and non-linear. % Would be good to add a sentence here about the main point of the paper, e.g. We show that XXXX and XXXX. Our results indicate that recent theory suggesting a trade-off in sensitivity between environmental cues needs to be revisited.
\vspace{-0.5ex}\\
Plant phenology plays a crucial role in ecosystem processes and is one of the most reported indicators of climate change. Yet as the wealth of observational data highlighting this rapid advance in phenology has increased, research has uncovered variation in these shifts across space, time and species. Understanding this variation has led to a number of studies and debates about how the major cues known to underlie phenology---spring forcing temperatures, winter chilling temperatures, and photoperiod---vary across species, and whether they may interact. Observational studies have highlighted that a simple model of temperature forcing cannot predict the observed variation, but have been hampered from providing further insights because the three major cues generally covary in nature. Advances in our understanding therefore require an experimental approach that manipulates all three cues across a community of species. 
\vspace{1.5ex}\\
\emph{What hypotheses or questions does this work address?} We test the prevalence of the three major cues known to drive temperate spring phenology---forcing temperatures, chilling temperatures, and photoperiod---across 28 species from two North American forest communities at two latitudes. We hypothesized that each species would respond to 1-2 cues (e.g., photoperiod and chilling but not forcing).
\vspace{1.5ex}\\
\emph{How does this work advance our current understanding of plant science?} Contrary to hypotheses and recent work, we found all species responded to all cues. Responses to photoperiod and forcing temperature were often similar among species and showed no evidence that some species could be categorized as insensitive to any cue. 
\vspace{1.5ex}\\
\emph{Why is this work important and timely?} We present the first multi-species study to assess all cues through direct manipulation and thus provide novel results on how cues vary across species. Our results suggest that predicting the spring phenology of communities will be difficult as all species we studied could have complex, non-linear responses to future warming.
\vspace{1.5ex}\\
% We have suggested three possible reviewers (see comments section of online submission system). 
Both authors substantially contributed to this work and approved of this version for submission. The manuscript is approximately 4,500 words with 200 word summary, and three figures. It is not under consideration elsewhere. We hope that you will find it suitable for publication in \emph{New Phytologist}, and look forward to hearing from you.
\vspace{1.5ex}\\
Thank you for your consideration.
\vspace{1.5ex}\\
\noindent Sincerely,\\

 \includegraphics[width=0.35\textwidth]{/Users/Lizzie/Documents/Professional/Vitas/Signatures/SignatureLizzieSm.png} 
%\noindent Elizabeth M Wolkovich (on behalf of my co-authors)

\end{document}
\newpage
\noindent {\bf References:}
\vspace{-5ex}
\bibliographystyle{/Users/Lizzie/Documents/EndnoteRelated/Bibtex/styles/naturemag}
\renewcommand{\refname}{\CHead{}}
\bibliography{/Users/Lizzie/Documents/git/projects/treegarden/budexperiments/docs/ms/danlib}

Understanding the cues that control spring leafout in forest communities is critical to accurate predictions of future growing seasons, plant communities, and a suite of related ecosystem services, including carbon sequestration. As such a growing body of literature has ...
Accurate predictions of spring plant phenology with continued climate change are critical for robust projections of growing seasons, plant communities and a suite of ecosystem services. 