\documentclass[11pt,a4paper]{letter}
\usepackage[top=1.00in, bottom=1.0in, left=1.1in, right=1.1in]{geometry}
\usepackage{graphicx}

%\signature{}

\begin{document}
\begin{letter}{}
\includegraphics[width=0.1\textwidth]{/Users/Lizzie/Documents/Professional/images/letterhead/ubc/UBClogo.jpg}

\opening{Dear Dr. Atkin:}
% While we could use trait databases or the like to search for correlations with our results, we do not believe this is the most elucidating path forward, and it is further not the focus of our manuscript. 

\noindent Please consider our revised manuscript `Temperature and photoperiod drive spring phenology across all species in a temperate forest community' as a Full Paper for \emph{New Phytologist.} 
\vspace{1.5ex}\\
Plant phenology plays a crucial role in ecosystem processes and is one of the most reported indicators of climate change. Yet as the wealth of observational data highlighting this rapid advance in phenology has increased, research has uncovered variation in these shifts across space, time and species. Such variation suggests many species may be responding to a suite of interactive cues. We test the prevalence of the three major cues known to drive temperate spring phenology---forcing temperatures, chilling temperatures, and photoperiod---across 28 species from two North American forest communities at two latitudes. Our results represent the first multi-species study to assess all cues through direct manipulation and thus provide novel results on how cues vary across species. % Our results suggest that predicting the spring phenology of communities will be difficult as all species we studied could have complex, non-linear responses to future warming.
\vspace{1.5ex}\\
Comments from three reviewers have greatly improved this manuscript and led us to present two major new analyses, including new data from a second year of experiments, alongside a much more nuanced discussion of our results. To address concerns regarding the rates of budburst and leafout success (Reviewer 2) we have provided new analyses (see \emph{Budburst and leafout success} in the Supporting Information, including new Tables S2-S4 and Figures S2-3), which find only minor effects of the treatments on success rates. We also clarified how our models account for varying sample sizes across species. Finally, we have re-run our main models using a reduced set of the data where budburst and leafout success were high and found our results are robust. For transparency our revised manuscript reviews these results in the main text. To address concerns regarding our results with respect to chilling (Reviewer 3) and whether our treatments resulted in insufficient chilling we now show that our chilling treatments generally exceed field chilling at both field sites and we provide data from a second year of experiments where we found similar chilling results (see \emph{Effects of chilling at 16 and 32 days} in the Supporting Information and Figure S9). We have also adjusted our discussion to more fully review the benefits and drawbacks of experimental chilling.
\vspace{1.5ex}\\
We have attempted to address all reviewer concerns and have added additional analyses to test the relationships among some cues as suggested by Reviewers 2 and 3.  We agree with Reviewer 2, who suggested that `[our reported] species-level responses could be correlated against many variables,' however we do not currently have the individual-level estimates we believe would be most appropriate for correlation with our estimates (and our model provides 14 estimates per species, requiring a thoughtful and careful modeling approach to any correlational analyses); we now review this in the discussion and better highlight how our results contribute to this area of research. We agree that this is an important area for future research, however it is not the focus of our manuscript. 
\vspace{1.5ex}\\
We feel the new submission is much improved and detail our changes in the following pages (note that reviewer comments are in \emph{italics}, while our responses are in regular text). The manuscript is approximately 5 020 words with 200 word summary, and three figures. It is not under consideration elsewhere. We hope that you will find it suitable for publication in \emph{New Phytologist}, and look forward to hearing from you.
\\\vspace{-1ex}\\
\noindent Sincerely,\\

 \includegraphics[width=0.3\textwidth]{/Users/Lizzie/Documents/Professional/Vitas/Signatures/SignatureLizzieSm.png} \\

\noindent Elizabeth M Wolkovich

\end{letter}
\end{document}



