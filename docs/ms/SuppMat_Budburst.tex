\documentclass{article}
\usepackage{textcomp}
\usepackage{fontenc}
\usepackage{graphicx}
\usepackage{caption}
\usepackage{gensymb} % for \degree
\usepackage{placeins} % for \images
\usepackage[margin=1in]{geometry} % to set margins
%\usepackage{cite} % for bibtex
\usepackage{natbib}

% \renewcommand{\familydefault}{\sfdefault}
\graphicspath{{images/}}	% Root directory of the figures
\setlength{\parskip}{2 mm}

% See IMPT NOTE re PNG figures! Below.

\usepackage{Sweave}
\begin{document}

 % 

\begin{center}
\textbf{\Large{Supporting Information \vspace{1ex}\\Temperature and photoperiod drive spring phenology across all species in a temperate forest community}}

Flynn \& Wolkovich
\end{center}

%% Add S prefix for tables and figures in Supplemental Materials
\renewcommand{\thetable}{S\arabic{table}}
\renewcommand{\thefigure}{S\arabic{figure}}

%%%%%%%%%%%%%%%%%%%%%%%%%%%%%%%
\section*{Supplemental Methods}
%%%%%%%%%%%%%%%%%%%%%%%%%%%%%%%

\noindent\emph{Temperature and photoperiod variation at study sites}

\noindent Spring climate regimes between our two sites (Harvard Forest and St. Hippolyte) vary, with temperatures generally being colder in the spring at the further north site (St. Hippolyte). Considering daily temperature data between 2000-2015 at both sites, daily minima, averaged over January-March each year, span -11.96 to -4.07\degree C and -19.74 to -8.88\degree C; daily maxima over the same period span -1.53 to 6.13\degree C and -9.99 to -0.26\degree C, at Harvard Forest and St. Hippolyte, respectively. Considering the period of April-May, daily minima span 2.90 to 6.19\degree C and 0.62 to 4.14\degree C; daily maxima over the same period span 14.23 to 19.00\degree C and 11.28 to 15.83\degree C at Harvard Forest and St. Hippolyte, respectively. 

\noindent As Saint Hippolyte is further north, its daylength (photoperiod) change in the spring is more extreme than at Harvard Forest: change from 1 March to 15 May is approximately 4 hours, while a similar change in Harvard Forest occurs over 1 March to 31 May. 

\noindent\emph{Chilling calculations}

\noindent The cuttings were harvested in late January 2015, and thus experienced substantial natural chilling by the time they were harvested. Using weather station data from the Harvard Forest and St. Hippolyte site, chilling hours (below 7.2\degree C), Utah Model chill portions \citep{utahmodel}, and Dynamic Model \citep{Erez:1988} chill portions were calculated both for the natural chilling experienced by harvest and the chilling experienced in the 4\degree C and 1.5\degree C treatments. The Utah Model and Dynamic Model of chill portions account for variation in the amount of chilling accumulated at different temperatures, with the greatest chilling occurring approximately between 5-10\degree C, and fewer chill portions accumulating at low temperatures and that higher temperatures can reduce accumulated chilling effects. The two differ in the parameters used to determine the shape of the chilling accumulation curve, with the Dynamic Model being shown to be the most successful in predicting phenology for some woody species \citep{Luedeling:2009}.
With both the Utah and Dynamic model, the more severe chilling treatment resulted in fewer calculated chilling portions. 

\noindent\emph{Non-leafouts}

\noindent Across all treatments, 20.2\% of the cuttings did not break bud or leaf out. Across species, there was no overall predictive effect of temperature, photoperiod, chilling, or site on the propensity to fail to leaf out. Species ranged from complete leafout (\emph{Hamamaelis}) to only 50\% leafout (\emph{Fagus grandifolia}, \emph{Acer saccharum}) across all treatments. The percent of non-leafouts by site was similar, with 20.6\% of Harvard Forest and 19.7\% of St. Hippolyte samples failing to leafout. Examining individual species,  there was an interaction of temperature by day length for selected species, with greater failure to leafout in cool, short-day conditions for \emph{Acer pensylvanicum}  and \emph{Acer saccharum}. Site effects were inconsistent, with greater failure to leafout for cuttings from St. Hippolyte in \emph{Acer rubrum} and \emph{Fagus grandifolia}, and from Harvard Forest in \emph{Acer saccharum}. 

% http://andrewgelman.com/2016/11/05/why-i-prefer-50-to-95-intervals/

%%%%%%%%%%%%%%%%%%%%%%%%%%%%%%%%%%%%
% \subsection*{References Cited}
%%%%%%%%%%%%%%%%%%%%%%%%%%%%%%%%%%%%

\bibliography{danlib}
\bibliographystyle{newphyto}

\newpage
%%%%%%%%%%%%%%%%%%%%%%%%%%%%%%%
\section*{Supplemental Figures and Tables}
%%%%%%%%%%%%%%%%%%%%%%%%%%%%%%%

% latex table generated in R 3.3.3 by xtable 1.8-2 package
% Fri Feb 16 16:06:06 2018
\begin{table}[ht]
\centering
\caption{Mean leafout and budburst days after exposure to controlled environment forcing conditions (across all treatments, based on raw data) for the 28 species at both Harvard Forest (HF), USA and St. Hippoltye (SH), Canada} 
\begin{tabular}{lrrrr}
  \hline
Species & Budburst.HF & Budburst.SH & Leafout.HF & Leafout.SH \\ 
  \hline
\textit{Acer pensylvanicum} & 16.40 & 18.33 & 40.88 & 46.94 \\ 
  \textit{Acer rubrum} & 22.40 & 25.15 & 40.59 & 44.40 \\ 
  \textit{Acer saccharum} & 44.96 & 36.48 & 57.07 & 46.88 \\ 
  \textit{Alnus incana subsp. rugosa} & 32.91 & 25.36 & 45.15 & 44.36 \\ 
  \textit{Aronia melanocarpa} & 13.62 &  & 29.83 &  \\ 
  \textit{Betula alleghaniensis} & 19.67 & 20.77 & 33.51 & 34.64 \\ 
  \textit{Betula lenta} & 29.83 &  & 50.57 &  \\ 
  \textit{Betula papyrifera} & 16.89 & 18.04 & 28.71 & 35.63 \\ 
  \textit{Corylus cornuta} & 24.86 & 19.04 & 33.95 & 30.38 \\ 
  \textit{Fagus grandifolia} & 41.82 & 43.13 & 48.54 & 46.90 \\ 
  \textit{Fraxinus nigra} & 38.00 & 38.00 & 52.28 & 46.91 \\ 
  \textit{Hamamelis virginiana} & 43.67 &  & 47.38 &  \\ 
  \textit{Ilex mucronatus} & 15.80 & 15.49 & 26.97 & 25.15 \\ 
  \textit{Kalmia angustifolia} & 30.25 & 32.48 & 37.80 & 42.20 \\ 
  \textit{Lonicera canadensis} & 16.91 & 15.75 & 28.26 & 25.08 \\ 
  \textit{Lyonia ligustrina} & 30.87 &  & 49.50 &  \\ 
  \textit{Nyssa sylvatica} & 31.65 &  & 52.87 &  \\ 
  \textit{Populus grandidentata} & 33.43 & 31.23 & 46.21 & 45.17 \\ 
  \textit{Prunus pensylvanica} & 17.81 & 16.21 & 32.13 & 29.65 \\ 
  \textit{Quercus alba} & 45.23 &  & 52.91 &  \\ 
  \textit{Quercus rubra} & 36.43 & 33.57 & 45.02 & 42.80 \\ 
  \textit{Quercus velutina} & 52.09 &  & 59.16 &  \\ 
  \textit{Rhamnus frangula} & 32.38 &  & 37.29 &  \\ 
  \textit{Rhododendron prinophyllum} & 29.25 &  & 52.14 &  \\ 
  \textit{Spiraea alba} & 18.00 & 20.21 & 25.94 & 24.62 \\ 
  \textit{Vaccinium myrtilloides} & 13.12 & 17.27 & 27.00 & 28.95 \\ 
  \textit{Viburnum cassinoides} & 15.41 & 18.46 & 16.80 & 18.71 \\ 
  \textit{Viburnum lantanoides} & 31.25 & 27.54 & 32.02 & 26.41 \\ 
   \hline
\end{tabular}
\end{table}
\clearpage
\newpage

% latex table generated in R 3.3.3 by xtable 1.8-2 package
% Fri Feb 16 16:06:06 2018
\begin{table}[ht]
\centering
\caption{Summary of mixed-effects model of budburst day. See also Figure S2.} 
\begin{tabular}{rrrrrrr}
  \hline
 & mean & sd & 2.5\% & 50\% & 97.5\% & Rhat \\ 
  \hline
Forcing Temperature & -8.82 & 1.05 & -10.87 & -8.82 & -6.73 & 1.00 \\ 
  Photoperiod & -4.53 & 0.90 & -6.28 & -4.53 & -2.73 & 1.00 \\ 
  Chilling 4 \degree C & -15.82 & 2.13 & -20.06 & -15.82 & -11.59 & 1.00 \\ 
  Chilling 1.5 \degree C & -13.13 & 2.00 & -17.04 & -13.14 & -9.08 & 1.00 \\ 
  Site & 1.31 & 1.08 & -0.84 & 1.31 & 3.45 & 1.00 \\ 
  Forcing Temperature $\times$ Photoperiod & -0.62 & 0.79 & -2.14 & -0.63 & 0.95 & 1.00 \\ 
  Forcing Temperature $\times$ Chilling 4 \degree C & 9.09 & 1.09 & 6.94 & 9.09 & 11.23 & 1.00 \\ 
  Forcing Temperature $\times$ Chilling 1.5 \degree C & 9.78 & 1.17 & 7.50 & 9.79 & 12.07 & 1.00 \\ 
  Photoperiod $\times$ Chilling 4 \degree C & -0.26 & 1.11 & -2.48 & -0.26 & 1.98 & 1.00 \\ 
  Photoperiod $\times$ Chilling 1.5 \degree C & -0.14 & 1.25 & -2.63 & -0.13 & 2.27 & 1.00 \\ 
  Forcing Temperature $\times$ Site & -1.51 & 0.85 & -3.13 & -1.51 & 0.16 & 1.00 \\ 
  Photoperiod $\times$ Site & -0.08 & 0.81 & -1.70 & -0.08 & 1.49 & 1.00 \\ 
  Site $\times$ Chilling 4 \degree C & -2.26 & 1.21 & -4.64 & -2.25 & 0.13 & 1.00 \\ 
  Site $\times$ Chilling 1.5 \degree C & -3.47 & 1.34 & -6.10 & -3.48 & -0.85 & 1.00 \\ 
   \hline
\end{tabular}
\end{table}
% latex table generated in R 3.3.3 by xtable 1.8-2 package
% Fri Feb 16 16:06:06 2018
\begin{table}[ht]
\centering
\caption{Summary of mixed-effects model of leafout day. See also Figure S3.} 
\begin{tabular}{rrrrrrr}
  \hline
 & mean & sd & 2.5\% & 50\% & 97.5\% & Rhat \\ 
  \hline
Forcing Temperature & -19.06 & 1.04 & -21.10 & -19.05 & -17.05 & 1.00 \\ 
  Photoperiod & -11.19 & 0.86 & -12.91 & -11.18 & -9.53 & 1.00 \\ 
  Chilling 4 \degree C & -17.44 & 2.07 & -21.55 & -17.42 & -13.34 & 1.00 \\ 
  Chilling 1.5 \degree C & -15.84 & 2.05 & -20.02 & -15.80 & -11.88 & 1.00 \\ 
  Site & 1.34 & 1.24 & -1.12 & 1.35 & 3.76 & 1.00 \\ 
  Forcing Temperature $\times$ Photoperiod & 3.68 & 0.85 & 2.06 & 3.66 & 5.39 & 1.00 \\ 
  Forcing Temperature $\times$ Chilling 4 \degree C & 10.30 & 1.17 & 8.01 & 10.30 & 12.62 & 1.00 \\ 
  Forcing Temperature $\times$ Chilling 1.5 \degree C & 11.22 & 1.32 & 8.61 & 11.24 & 13.81 & 1.00 \\ 
  Photoperiod $\times$ Chilling 4 \degree C & 0.79 & 1.18 & -1.47 & 0.79 & 3.16 & 1.00 \\ 
  Photoperiod $\times$ Chilling 1.5 \degree C & 2.35 & 1.32 & -0.32 & 2.37 & 4.87 & 1.00 \\ 
  Forcing Temperature $\times$ Site & -0.50 & 0.83 & -2.11 & -0.50 & 1.15 & 1.00 \\ 
  Photoperiod $\times$ Site & -0.87 & 0.83 & -2.51 & -0.87 & 0.75 & 1.00 \\ 
  Site $\times$ Chilling 4 \degree C & -1.75 & 1.28 & -4.26 & -1.76 & 0.77 & 1.00 \\ 
  Site $\times$ Chilling 1.5 \degree C & -3.35 & 1.51 & -6.35 & -3.34 & -0.41 & 1.00 \\ 
   \hline
\end{tabular}
\end{table}
\newpage

% latex table generated in R 3.3.3 by xtable 1.8-2 package
% Fri Feb 16 16:06:06 2018
\begin{table}[ht]
\centering
\caption{Chill units in the field before harvest, until 1 April and 1 May at each site, and in growth chamber conditions (including field chilling experienced before cuttings entered the chambers).} 
\begin{tabular}{llrrr}
  \hline
Site & Treatment & Chilling Hours & Utah Model & Chill portions \\ 
  \hline
Harvard Forest & Field chilling until collection & 892 & 814.50 & 56.62 \\ 
   & Field chilling to 1 Apr & 1153 & 980.00 & 79.37 \\ 
   & Field chilling to 1 May & 1449 & 1360.50 & 99.94 \\ 
   & Field chilling + 4.0 \degree C x 30 d & 2140 & 2062.50 & 94.06 \\ 
   & Field chilling + 1.5 \degree C x 30 d & 2140 & 1702.50 & 91.17 \\ 
  St. Hippolyte & Field chilling until collection & 682 & 599.50 & 44.63 \\ 
   & Field chilling to 1 Apr & 796 & 653.50 & 63.55 \\ 
   & Field chilling to 1 May & 1166 & 937.00 & 83.83 \\ 
   & Field chilling + 4.0 \degree C x 30 d & 1930 & 1847.50 & 82.06 \\ 
   & Field chilling + 1.5 \degree C x 30 d & 1930 & 1487.50 & 79.18 \\ 
   \hline
\end{tabular}
\end{table}

%% \clearpage % use if get 'too many unprocessed floats' error

\clearpage 
% IMPT NOTE re PNG figures: Annoyingly, Sweave seems to delete the degree marks from the Advplot2 and 4 panel figs. They run fine if output from Pheno Budburst Analysis.R ... I could fix this I am sure, but for cheap sake, I just changed them to PNG. This is important to remember if the data or analysis ever changes!!

% Fig S1: Raw data plot

\begin{figure} 
\begin{center}
\includegraphics[scale=0.5]{Advplot2.png}
\caption{Responses of 28 woody plant species to photoperiod and temperature cues for leafout. Color of circle reflect unmodeled data on average leafout day across treatments, across sites of origin, while size of circle represents the total number of clippings in the experiment---this varies mainly based on whether the species was found at both sites and whether it was exposed to all three chilling treatments. } % Changed legend, check that I did it correctly please! Yes, ok.
\label{fig1}
\end{center}
\end{figure}

\begin{figure}
\label{figS8}
\includegraphics[width=1\textwidth, page=1]{NonBBLO_sp}
\caption{Model estimates of budburst success, including species-level effects. Dots and bars show mean and 50\% credible intervals.}
\end{figure}

\begin{figure}
\label{figS8}
\includegraphics[width=1\textwidth, page=2]{NonBBLO_sp}
\caption{Model estimates of leafout success, including species-level effects. Dots and bars show mean and 50\% credible intervals.}
\end{figure}

\begin{figure}
\label{figS2}
\includegraphics[width=1\textwidth, page=1]{Fig1_bb_lo+sp} % built in Additional Plots and Processing.R
\caption{Model estimates of effects of each predictor on days to budburst, including species-level effects. Dots and bars show mean and 50\% credible intervals.}
\end{figure}

\clearpage

\begin{figure}
\label{figS3}
\includegraphics[width=1\textwidth, page=2]{Fig1_bb_lo+sp}
\caption{Model estimates of days to leafout leafout, including species-level effects. Dots and bars show mean and 50\% credible intervals.}
\end{figure}

\begin{figure}
\label{figS5}
\includegraphics[width=1\textwidth]{Fig2_4panel_ZoomSupp.png}
\caption{Effects of photoperiod, temperature and chilling across species: Similar to Fig. 2 from main text, but designed to make species names easier to read by adjusting axes (note that axes vary across rows and columns) and removing bars showing credible intervals. }
\end{figure}


\begin{figure}
\label{figS6}
\includegraphics[width=1\textwidth]{FigChill2_4panel.png}
\caption{Effects of temperature and chilling across species: Similar to Fig. 2 from main text, but showing results side-by-side for the two chilling treatments: 4.0\degree C (left) versus 1.5\degree C (right) as compared to forcing temperature responses (for photoperiod see S7). Crosses and bars show mean and 50\% credible intervals.}
\end{figure}


\begin{figure}
\label{figS7}
\includegraphics[width=1\textwidth]{FigChillPhoto_4panel.png}
\caption{Effects of photoperiod and chilling across species: Similar to Fig. 2 from main text, but showing results side-by-side for the two chilling treatments: 4.0\degree C (left) versus 1.5\degree C (right) as compared to photoperiod responses (for forcing temperature see S6). Crosses and bars show mean and 50\% credible intervals.}
\end{figure}

\end{document}


\noindent\emph{Literature Review} (Note: I am not sure we need this anymore.)

\noindent We conducted a literature review, finding 109 studies which investigated effects of photoperiod, temperature, or their interaction on the timing of bud burst or flowering for woody or semi-woody plants.  No study varied chilling period, photoperiod, and temperature simultaneously across multiple species at multiple sites. Of those studies, eight simultaneously manipulated photoperiod and temperature. \citet{Basler:2014aa} found a negative tradeoff between sensitivity to photoperiod and sensitivity to warming for four species, for example with \emph{Fagus sylvatica} advanced on average in leafout by 12 days in response to experimentally lengthened photoperiod, but only ca. 8 days in response to warmer temperatures, while \emph{Acer pseudoplatanus} advanced in leafout by 17 days in response to warming but essentially had no change in response to photoperiod. The current study expands on this work by including 28 species, across two sites, with addition manipulations of chilling temperature.


%% Trait and HF (O'Keefe) plots
% Used to go before end{document} 
\clearpage

\begin{figure}
\caption{Trait sensitivity based on specific leaf area}
\label{figS5}
\includegraphics[scale=0.95, page=1]{Traits_vs_sensitivity}
\end{figure}

\begin{figure}
\caption{Trait sensitivity based on stem density}
\label{figS6}
\includegraphics[scale=0.95, page=2]{Traits_vs_sensitivity}
\end{figure}

\begin{figure}
\caption{Trait sensitivity based on \% nitrogen}
\label{figS7}
\includegraphics[scale=0.95, page=3]{Traits_vs_sensitivity}
\end{figure}


\begin{figure}
\caption{Specific leaf area and stem density by trees vs shrubs}
\label{figS8}
\includegraphics[scale=0.95, page=1]{Tree_shrub_traits}
\end{figure}


\begin{figure}
\caption{Specific leaf area and percent nitrogen by trees vs shrubs}
\label{figS9}
\includegraphics[scale=0.95, page=2]{Tree_shrub_traits}
\end{figure}

\begin{figure}
\caption{Stem density and percent nitrogen by trees vs shrubs}
\label{figS10}
\includegraphics[scale=0.95, page=3]{Tree_shrub_traits}
\end{figure}

\clearpage


\begin{figure}
\caption{Leafout rank order in experimental treatments vs. O'Keefe observations}
\label{figS11}
\includegraphics{leafout_exp_obs_corr}
\end{figure}

\begin{figure}
\caption{Leafout day of year in experimental treatments vs. O'Keefe observations}
\label{figS12}
\includegraphics{leafout_exp_obs_corr_day}
\end{figure}

 
